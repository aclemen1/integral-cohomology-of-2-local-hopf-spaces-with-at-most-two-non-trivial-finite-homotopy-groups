\documentclass{amsart}

%\usepackage{amsthm,amssymb,amsfonts,amsxtra}
\usepackage{amssymb,amsfonts,amsxtra}
%\usepackage{ifthen}
\usepackage[all]{xy}
%\usepackage{graphicx}
%\usepackage{algorithm}
%\usepackage{algorithmic}
\usepackage{moreverb}
\usepackage{supertabular}
\usepackage{multirow}
\usepackage{picinpar}
%\usepackage{fancybox}
%\usepackage[first,light,english]{draftcopy}

\newtheorem{thm}{Theorem}[section]
\newtheorem*{thm*}{Theorem}
\newtheorem{lem}[thm]{Lemma}
\newtheorem{prop}[thm]{Proposition}
\newtheorem*{prop*}{Proposition}
\newtheorem{cor}[thm]{Corollary}
\newtheorem*{cor*}{Corollary}

\theoremstyle{definition}
\newtheorem{defn}[thm]{Definition}
\newtheorem*{defn*}{Definition}
\newtheorem{nota}[thm]{Notation}
\newtheorem{conv}[thm]{Convention}
\newtheorem*{conv*}{Convention}
\newtheorem{conj}[thm]{Conjecture}
\newtheorem*{conj*}{Conjecture}
\newtheorem{quest}[thm]{Question}
\newtheorem*{quest*}{Question}
\newtheorem{exmp}[thm]{Example}
\newtheorem*{exmp*}{Example}
\newtheorem{alg}[thm]{Algorithm}

\theoremstyle{remark}
\newtheorem{rem}[thm]{Remark}
\newtheorem*{rem*}{Remark}
\newtheorem*{note*}{Note}

%\numberwithin{section}{section}
%\numberwithin{equation}{subsection}

\DeclareMathOperator{\id}{id}
\DeclareMathOperator{\proj}{proj}
\DeclareMathOperator{\incl}{incl}
\DeclareMathOperator{\FGC}{FGC}
\DeclareMathOperator{\Hom}{Hom}
\DeclareMathOperator{\map}{map}
\DeclareMathOperator{\red}{red}
\DeclareMathOperator{\rk}{rank}
\DeclareMathOperator{\subker}{sub-ker}
\DeclareMathOperator{\im}{im}
\DeclareMathOperator{\odd}{^{odd}}
\DeclareMathOperator{\even}{^{even}}
\DeclareMathOperator{\teven}{even}
\DeclareMathOperator{\deven}{_{even}}
\DeclareMathOperator{\Tor}{Tor}
\DeclareMathOperator{\bideg}{bideg}
\DeclareMathOperator{\degst}{deg_{st}}
\DeclareMathOperator{\card}{Card}
\DeclareMathOperator{\dual}{dual}
\DeclareMathOperator{\coker}{coker}
\newcommand{\N}{\mathbb{N}}
\newcommand{\Z}{\mathbb{Z}}
\newcommand{\Q}{\mathbb{Q}}
\newcommand{\R}{\mathbb{R}}
\newcommand{\F}{\mathbb{F}}
\newcommand{\A}{\mathcal{A}}
\newcommand{\clearemptydoublepage}{\newpage{\pagestyle{empty}\cleardoublepage}}
\renewcommand{\geq}{\geqslant}
\renewcommand{\leq}{\leqslant}

%\textwidth = 6.5 in
%\textheight = 9 in
%\oddsidemargin = 0.0 in
%\evensidemargin = 0.0 in
%\topmargin = 0.0 in
%\headheight = 0.0 in
%\headsep = 0.0 in
%\parskip = 0.2in
%\parindent = 0.0in

\begin{document}

\title{Integral Cohomology Of Spaces With At Most Two Non-trivial Homotopy Groups}

\author{Alain Cl\'ement}
%\address{Institute of Mathematics, Faculty of Sciences, University of Lausanne, Switzerland}
%\curraddr{Institute of Mathematics, Faculty of Sciences, University of Lausanne, BCH 4103, CH-1015 Lausanne-Dorigny, Switzerland}
%\email{alain.clement@ima.unil.ch}
% ;o)
\thanks{The author was supported in part by the Swiss National Fund Grant.}

\date{\today}

% ;o)
% \subjclass[2000]{Primary 54C40, 14E20;\\Secondary 46E25, 20C20}

% ;o)
% \keywords{}

\begin{abstract}
In this paper we study one general aspect of the integral cohomology of a space $X$, namely the existence of an exponent $e\geq1$ such that $e\cdot \widetilde{H}^*(X;\Z)=0$. Our main result states that a non-contractible simply-connected $2$-local H-space with at most two non-trivial finite homotopy groups has no such exponent.
\end{abstract}

\maketitle

%\tableofcontents
%\cleardoublepage
%\clearpage

\section{Introduction}

Let $X$ be a connected space. One can both consider its graded homotopy group $\pi_*(X)$ and its graded reduced integral cohomology group $\widetilde{H}^*(X;\Z)$. If there exists an integer $h\geq1$ such that $h\cdot\pi_*(X)=0$, we then say that $X$ has a {\it homotopy exponent}. Analogously, if there exists an integer $e\geq1$ such that $e\cdot \widetilde{H}^*(X;\Z)=0$, we then say that $X$ admits a {\it homology exponent}.

A general problem posed by my thesis advisor D. Arlettaz suggests to explore the relationships between homotopy exponents and homology exponents, if there are. For instance, is it true that a space with a homotopy exponent has a homology exponent, too? In this case, how are these two exponents related? Or conversely, is it possible for a space without a homotopy exponent to admit a homology exponent?

In this paper, we focus on simply-connected $2$-local H-spaces with one or two non-trivial finite homotopy groups. These spaces obviously admit homotopy exponents. Our main result is the following:

\begin{thm*}
Let $X$ be a non-contractible simply-connected $2$-local H-space with at most two non-trivial finite homotopy groups. Then $X$ has no homology exponent.
\end{thm*}

The result when $X$ is an Eilenberg-Mac\,Lane space is an obvious consequence resulting from the calculations of H. Cartan \cite{Ca55}, as it is shown in Section \ref{s:transverse}. More elaborated techniques are required to prove the result when $X$ has two non-trivial homotopy groups. Some interesting examples are completely carried out in Section \ref{s:examples} and a proof is given in Section \ref{s:proofs}. We conclude the paper with some questions and comments in Section \ref{s:generalizations}

\subsection*{Acknowledgements}

Foremost I would like to thank my thesis advisor D. Arlettaz for the invaluable help that he provided me all along my studies and for the suggestions brought in this paper. I am also indebted to J. Scherer and C. Casacuberta for their support during my thesis work.

\section{Recollections}\label{s:recollections}

In this introductory section we collect the well-known results we need later in the paper.

Unless otherwise specified, a {\bf space} will mean a pointed, connected and simple topological space with the homotopy type of a CW-complex of finite type. 

We will denote by $K(G,n)$ the {\bf Eilenberg-Mac\,Lane space}\index{Eilenberg-Mac\,Lane space} with single non-trivial homotopy group isomorphic to $G$ in dimension $n$ ($G$~abelian if $n\geq2$).

Since we will only consider simple spaces, we will only deal with abelian fundamental groups and it will always be possible to consider the {\bf Postnikov tower}\index{Postnikov tower} and the {\bf k-invariants}\index{k-invariants}\index{Postnikov invariants} of a space.  Let us recall that the Postnikov tower of a space $X$ is given by the following homotopy commutative diagram:
$$\xymatrix{
&&\\
&&X[n+1] \ar@{.}[u] \ar[d]^{i_n} &K(\pi_{n+1}X,n+1) \ar[l] \\
X\ar[rru]^{\alpha_{n+1}}
\ar[rr]^{\alpha_n}
\ar[rrd]^{\alpha_{n-1}}
\ar[rrdd]_{\alpha_1}
&&X[n] \ar[d]^{i_{n-1}} &K(\pi_{n}X,n) \ar[l] \\
&&X[n-1] \ar@{.}[d] &K(\pi_{n-1}X,n-1) \ar[l] \\
&&X[1] &K(\pi_1 X,1), \ar[l]_\simeq
}$$ where $\alpha_n:X\to X[n]$ is the $n$-th {\bf Postnikov section}\index{Postnikov section} and $X[n]$ is given by the homotopy pullback along the $(n+1)$-th k-invariant $k^{n+1}(X)\in H^{n+1}(X[n-1];\pi_n X)\cong[X[n-1],K(\pi_n X,n+1)]$ and the path-loop fibration over $K(\pi_n X,n+1)$:
$$\xymatrix{
X[n]\ar[r]\ar[d]_{i_{n-1}} &PK(\pi_n X,n+1)\simeq{*}\ar[d]\\
X[n-1]\ar[r]_-{k^{n+1}(X)} &K(\pi_n X,n+1).
}$$

Let $X$ be a space. Its {\bf mod-$2$ cohomology Bockstein spectral sequence}\index{Bockstein spectral sequence} $\{B^*_r,d_r\}$ is the spectral sequence given by the exact couple 
$$\xymatrix@C=0.4truecm{
H^*(X;\Z)\ar[rr]^-{(\cdot2)_*} &&H^*(X;\Z)\ar[ld]^{(\red_2)_*}\\
&H^*(X;\F_2)\ar[lu]^{\partial}
}$$
which is the long exact sequence in cohomology induced by the short exact sequence of coefficients
$$\xymatrix{
0\ar[r] &\Z\ar[r]^-{\cdot2} &\Z\ar[r]^-{\red_2} &\Z/2\ar[r] &0.
}$$
The first page $B_1^*$ is isomorphic to $H^*(X;\F_2)$ and the first differential $d_1=(\red_2)_*\partial=\beta$ is the {\bf Bockstein homomorphism}\index{Bockstein homomorphism}. 

The spectral sequence converges to $(H^*(X;\Z)/\text{torsion})\otimes\F_2$. Moreover, if $x\in B_r^{n}$ is such that $d_r x\not=0\in B_r^{n+1}$, then $d_r x$ detects an element of order $2^r$ in $H^{n+1}(X;\Z)$, i.e. there exists an element $y$ of order $2^r$ in $H^{n+1}(X;\Z)$ such that $\red_2(y)=d_rx$. 

Let $H$ be a {\bf graded Hopf algebra}\index{Hopf algebra} over the field $\F_2$, with multiplication $\mu:H\otimes H\to H$, comultiplication $\Delta:H\to H\otimes H$, augmentation $\epsilon:H\to\F_2$ and unit $\eta:\F_2\to H$ ($\F_2$ is concentrated in degree zero, see \cite{MM65} for the definitions). The {\bf augmentation ideal}\index{augmentation ideal} of $H$ is denoted by
$$
\bar{H}=\ker\epsilon:H\to\F_2.
$$ We will denote  by
\begin{align*}
QH&=\bar{H}/\mu(\bar{H}\otimes\bar{H})\\
&=\coker \mu:\bar{H}\to\bar{H}\otimes\bar{H}
\end{align*}
the graded module of {\bf indecomposable}\index{indecomposable}\index{module of indecomposables} elements of $H$. We will denote
\begin{align*}
PH&=\{x\in\bar{H}\ |\ \Delta(x)=x\otimes1+1\otimes x\}\\
&=\ker \Delta:\bar{H}\to\bar{H}\otimes\bar{H}
\end{align*}
the graded module of {\bf primitive}\index{primitive}\index{module of primitives} elements of $H$.  

The {\bf Milnor-Moore theorem}\index{Milnor-Moore theorem}\label{t:Milnor-Moore} states that there is an exact sequence of graded modules
$\xymatrix{
0\ar[r] &P(\xi H)\ar[r] &PH\ar[r] &QH,
}$ where $\xi H$ is the image of the {\bf Frobenius map}\index{Frobenius map} $\xi:x\mapsto x^2$. The Hopf algebra $H$ is said to be {\bf primitively generated} if $PH\to QH$ in the above exact sequence is an epimorphism.

A non-empty finite sequence of positive integers $I=(a_0,\dots,a_k)$, where $k$ is varying, is {\bf admissible} if $a_i\geq2a_{i+1}$ for all $0\leq i\leq k-1$. Let $\mathcal S$ be the set of all such admissible sequences. The {\bf stable degree} is a map $\degst:{\mathcal S}\to\N$ defined by $\degst(I)=\sum_{i=0}^k a_i$ for all $I=(a_0,\dots,a_k)\in{\mathcal S}$. The stable degree induces a {\bf grading} on the set $\mathcal S$ of all admissible sequences. The {\bf excess} is a map $e:{\mathcal S}\to\N$ defined by $e(I)=2a_0-\degst(I)=a_0-\sum_{i=1}^k a_i$ for all $I=(a_0,\dots,a_k)\in{\mathcal S}$.

Let $n\geq1$ and $s\geq1$. Let $\delta_s$ the connecting homomorphism in the long exact sequence of coefficients in cohomology associated to the short exact sequence
$$\xymatrix{0\ar[r] &\Z/2\ar[r] &\Z/2^{s+1}\ar[r] &\Z/2^s\ar[r] &0}.$$ 
Consider the fundamental class $\iota_n\in H^n(K(\Z/2^s,n);\Z/2^s)$ and its mod-$2$ reduction $u_n\in H^n(K(\Z/2^s,n);\F_2)$. Let $I=(a_0,\dots,a_k)$ be an admissible sequence.
\begin{conv*}
We will write $Sq^I_s u_n$ instead of $Sq^{a_0}\dots Sq^{a_{k-1}}\delta_s\iota_n$ (usually denoted by $Sq^{a_0,\dots,a_{k-1}}\delta_s\iota_n$) if $a_k=1$ and instead of $Sq^{a_0}\dots Sq^{a_k} u_n$ (also denoted by $Sq^{a_0,\dots,a_k} u_n$ or $Sq^I u_n$) if $a_k\not=1$. In particular, since $\delta_1=Sq^1$ and the reduction is the identity when $s=1$, we have $Sq^I_1 u_n=Sq^I u_n$.
\end{conv*}

For any $n\geq0$, the cohomology group $H^n(X;\F_2)$ is a $\F_2$-vector space. These objects constitute a graded $\F_2$-vector space which we denote by $H^*(X;\F_2)$. Moreover, $H^*(X;\F_2)$, endowed with the cup product (denoted by $\cdot$), is a graded $\F_2$-algebra.

\medskip
\newpage
J.-P. Serre \cite{Se53} computed the mod-$2$ cohomology of Eilenberg-Mac\,Lane spaces and stated the following result:


\begin{thm*}
Let $n\geq1$ and $s\geq1$. The graded $\F_2$-algebra $H^*(K(\Z/2^s,n);\F_2)$ is isomorphic to the graded polynomial $\F_2$-algebra on generators $Sq^I_s u_n$, where $I$ covers all the admissible sequences of excess $e(I)<n$ and $u_n$ is the reduction of the fundamental class (see the above convention). The degree of a generator $Sq^I_s u_n$ is $\deg(Sq^I_s u_n)=\degst(I)+n$. 
\end{thm*}

%\begin{proof}
%See \cite[Th\'eor\`eme 2, p. 203 and Th\'eor\`eme 4, p. 206]{Se53}.
%\end{proof}

This result also reveals the $\A_2$-algebra structure of $H^*(K(\Z/2^s,n);\F_2)$, where $\A_2$ denotes the mod-$2$ Steenrod algebra. Recall that the products in $\A_2$ are related by the Adem relations:

\begin{align*}
Sq^iSq^j=\sum_{k=0}^{[i/2]}\binom{j-k-1}{i-2k}Sq^{i+j-k}Sq^k\\
\text{for all $0<i<2j$.}
\end{align*}

\begin{align*}
Sq^1Sq^1=0, Sq^1Sq^3=0,\dots; &&&Sq^1Sq^{2n+1}=0\\
Sq^1Sq^2=Sq^3,Sq^1Sq^4=Sq^5,\dots; &&&Sq^1Sq^{2n}=Sq^{2n+1}\\
Sq^2Sq^2=Sq^3Sq^1, Sq^2Sq^6=Sq^7Sq^1,\dots; &&&Sq^2Sq^{4n-2}=Sq^{4n-1}Sq^1\\
Sq^2Sq^3=Sq^5+Sq^4Sq^1,\dots; &&&Sq^2Sq^{4n-1}=Sq^{4n+1}+Sq^{4n}Sq^1\\
Sq^2Sq^4=Sq^6+Sq^5Sq^1,\dots; &&&Sq^{2}Sq^{4n}=Sq^{4n+2}+Sq^{4n+1}Sq^1\\
Sq^2Sq^5=Sq^6Sq^1,\dots; &&&Sq^2Sq^{4n+1}=Sq^{4n+2}Sq^1\\
Sq^3Sq^2=0, Sq^3Sq^6=0,\dots; &&&Sq^3Sq^{4n+2}=0\\
Sq^3Sq^3=Sq^5Sq^1,\dots; &&&\dots\\
\dots;&&&Sq^{2n-1}Sq^n=0\\
\vdots
\end{align*}

For instance, let us determine the unstable $\A_2$-algebra structure of the $\F_2$-algebra $H^*(K(\Z/2,1);\F_2)$. The excess of an admissible sequence $I$ is zero if and only if $I=(0)$ and the fundamental class $u_1\in H^1(K(\Z/2,1);\F_2)$ is then the only generator. Therefore we have
$$
H^*(K(\Z/2,1);\F_2)\cong \F_2[u_1]
$$ and the $\A_2$-action is given pictorially as follows
$$\xymatrix@R=-0.1cm{
\bullet &\bullet\ar@/^10pt/[r]^-{Sq^1} &\bullet\ar@/^20pt/[rr]^-{Sq^2} &\bullet\ar@/^10pt/[r]_-{Sq^1} &\bullet &\bullet\ar@/^10pt/[r]^-{Sq^1} &\bullet \\
1 &u_1 &u_1^2 &u_1^3\ar@/_10pt/[rr]_-{Sq^2}\ar@/_30pt/[rrr]_-{Sq^3} &u_1^4 &u_1^5 &u_1^6 &\dots
}$$

Let us also remark that the $\A_2$-action on $H^*(K(\Z/2^s,n);\F_2)$ provides a differential graded $\F_2$-algebra structure under the Bockstein homomorphism $\beta=Sq^1$.

It is well known that an Eilenberg-Mac\,Lane space associated with an abelian group has a unique H-space structure up to homotopy (which can be seen as inherited from the loop space structure or from the addition law of the associated abelian group). Therefore, the differential graded $\A_2$-algebra $H^*(K(\Z/2^s,n);\F_2)$ is also a differential graded Hopf algebra. J.-P. Serre also proved the following key result:

\begin{thm*}
Let $n\geq1$ and $s\geq1$. The differential graded $\A_2$-algebra $H^*(K(\Z/2^s,n);\F_2)$ is a connected, associative, commutative and primitively generated differential graded Hopf algebra.
\end{thm*}

%\begin{proof}
%See \cite[pp. 54-55]{Sm70}.
%\end{proof}

It is now easy to determine the modules of primitives and indecomposables of $H^*=H^*(K(\Z/2^s,n);\F_2)$. The module of indecomposable elements is clearly given by
\begin{align*}
QH^*&\cong\F_2\{Sq^I_s u_n\ |\ \text{$I$ admissible and $e(I)<n$}\},\\
&\text{the graded $\F_2$-vector space generated by}\\
&\text{all the $Sq^I_s u_n$ with $I$ admissible and $e(I)<n$.}\\
\end{align*}
Since $H^*$ is primitively generated, the Milnor-Moore theorem gives the following short exact sequence of graded $\F_2$-vector spaces:
$$\xymatrix{
0\ar[r] &P(\xi H)\ar[r] &PH\ar[r] &QH\ar[r] &0.
}$$ Therefore, every indecomposable element is primitive and every primitive element which is decomposable is a square of a primitive element. Thus we have
\begin{align*}
PH^*&\cong\F_2\{(Sq^I_s u_n)^{2^i}\ |\ \text{$I$ admissible, $e(I)<n$ and $i\geq0$}\},\\
&\text{the graded $\F_2$-vector space generated by}\\
&\text{all the iterated squares of $Sq^I_s u_n$ with $I$ admissible and $e(I)<n$.}\\
\end{align*}

\section{Transverse implications}\label{s:transverse}

In this section we present some concepts and results on the high torsion in the integral cohomology of Eilenberg-Mac\,Lane spaces associated to $2$-torsion groups of finite type. The material exposed here can be found with all the details in \cite{Cl02}. It is mainly inspired by the work of H. Cartan in \cite{Ca55}.

Let us start with the following key definition:

\begin{defn*}
Let $X$ be a space and $\{B^*_r,d_r\}$ be its mod-$2$ cohomology Bockstein spectral sequence. Let $n$ and $r$ be two positive integers. An element $x\in B^n_r$ is said to be {\bf $\ell$-transverse} if $d_{r+l}x^{2^l}\not=0\in B^{2^l n}_{r+l}$ for all $0\leq l\leq\ell$. An element $x\in B^n_r$ is said to be {\bf $\infty$-transverse}, or simply {\bf transverse}, if it is $\ell$-transverse for all $\ell\geq0$. We will also speak of {\bf transverse implications} of an element $x\in B^n_r$.
\end{defn*}

For instance, suppose that $x\in B_1^2$ is $\infty$-transverse and let us picture how the transverse implications of $x$ look like within the Bockstein spectral sequence:

$$\xymatrix@C=0.2cm@R=0.5truecm{
B_3^* &&\ar@{.}[rrrrrrrr] & & & & & & & &x^4\ar@/^0.4truecm/[r]^-{d_3} &\bullet\ar@{.}[rr] & &  &\dots\\
B_2^* &&\ar@{.}[rrrr] & & & &x^2\ar@/^0.4truecm/[r]^-{d_2} &\bullet\ar@{.}[rrrrrr] & & & & & &  &\dots\\
B_1^* &&\ar@{.}[rr] & &x\ar@/^0.4truecm/[r]^-{d_1} &\bullet\ar@{.}[rrrrrrrr] & & & & & & & & &\dots\\
{*}&&0 &1 &2 &3 &4 &5 &6 &7 &8 &9 &10 &11 &\dots
}$$

Every transverse element gives rise to $2$-torsion of arbitrarily high order in the integral cohomology of $X$. Our strategy for disproving the existence of a homology exponent for a space will then consist in exhibiting a transverse element in its mod-$2$ cohomology Bockstein spectral sequence. 

In our special case of Eilenberg-Mac\,Lane spaces, we have the following result:

\begin{thm*}
Let $G$ be a non-trivial $2$-torsion abelian group of finite type isomorphic to $\Z_{(2)}^{\times s}\Z/2^{s_1}\oplus\dots\oplus\Z/2^{s_l}$ and let $n\geq2$. Consider the Eilenberg-Mac\,Lane space $K(G,n)$ and its mod-$2$ cohomology Bockstein spectral sequence $\{B^*_r,d_r\}$. Suppose that one of the following assumptions holds:
\begin{itemize}
\item[$\bullet$]{$n$ is even and $x\in B^n_{s_j}$ is $0$-transverse for any $1\leq j\leq l$,}
\item[$\bullet$]{$x\in P\even B_1^*$ is $0$-transverse ($Sq^1x\not=0$).}
\end{itemize}
Then $x$ is $\infty$-transverse.
\end{thm*}

In \cite{Cl02}, an algorithm for explicitly computing the integral cohomology groups of such $K(G,n)$ spaces was implemented within a C++ program: the {\it Eilenberg-Mac\,Lane machine}. The machine produced the following table (truncated here) for $K(\Z/2,2)$: 

\medskip
\begin{tabular}{|c|p{3cm}|p{8cm}|}
\hline
$n$ &$H^n(K(\Z/2,2);\Z)$ &$H^n(K(\Z/2,2);\F_2)$ \\
\hline

$0$%
&$\Z$
&$\F_2$\\

$1$%
&$(0)$%
&$(0)$\\

$2$%
&$(0)$%
&$\F_2\{u_2\}$\\

$3$%
&$\Z/2$%
&$\F_2\{Sq^1u_2\}$\\

$4$%
&$(0)$%
&$\F_2\{u_2^2\}$\\

$5$%
&$\Z/2^{2}$%
&$\F_2\{Sq^{2,1}u_2,u_2Sq^1u_2\}$\\

$6$%
&$\Z/2$%
&$\F_2\{u_2^3,(Sq^1u_2)^2\}$\\

$7$%
&$\Z/2$%
&$\F_2\{u_2Sq^{2,1}u_2,u_2^2Sq^1u_2\}$\\

$8$%
&$\Z/2$%
&$\F_2\{u_2^4,u_2(Sq^1u_2)^2,Sq^1u_2Sq^{2,1}u_2\}$\\

$9$%
&$\Z/2\oplus\Z/2^{3}$%
&$\F_2\{Sq^{4,2,1}u_2,u_2^2Sq^{2,1}u_2,u_2^3Sq^1u_2,(Sq^1u_2)^3\}$\\

\hline
\end{tabular}
\medskip

The elements of order $2$, $4$ and $8$ in degrees $3$, $5$ and $9$ respectively are given by an $\infty$-transverse element: the characteristic class $u_2\in H^2(K(\Z/2,2);\F_2)$ -- which is of even degree and $0$-transverse -- and its iterated squares $u_2^2$ and $u_2^4$.

Let us remark that a $0$-transverse implication does not imply $\infty$-transverse implications for more general H-spaces. More precisely, the fact that $x\in P\even H^*(X;\F_2)$ is such that $Sq^1x\not=0$ does not always force $x$ to be $\infty$-transverse. A counterexample is given by $X=BSO$ and $x=w_2$, the second Stiefel-Withney class in $H^2(BSO;\F_2)$.

As a corollary of the previous theorem, it is then possible to give a proof of our main result for Eilenberg-Mac\,Lane spaces:

\begin{cor*}
Let $G$ be a non-trivial finite $2$-torsion abelian group and $n\geq2$. The Eilenberg-Mac\,Lane space $K(G,n)$ has no homology exponent.
\end{cor*}

\begin{proof}
According to the K\"unneth formula, it is sufficient to establish the result when $G=\Z/2^s$ for some $s\geq1$. If $n$ is even, consider the reduction of the fundamental class $u_n\in H^n(K(\Z/2^s,n);\F_2)$. This class survives to $B^n_s$ and is $0$-transverse. Then $u_n\in B^n_s$ is $\infty$-transverse. If $n$ is odd, consider the admissible sequence $(2,1)$. Its excess is exactly $1$ and therefore $Sq^{2,1}_s u_n\in P\even H^*(K(\Z/2,n);\F_2)$ when $n\geq3$. Moreover we have $Sq^1Sq^{2,1}_s u_n=Sq^{3,1}_s u_n$ by Adem relations, which means that $Sq^{2,1}_s u_n$ is $0$-transverse. Hence $Sq^{2,1}_s u_n\in B^{n+3}_1$ is $\infty$-transverse.
\end{proof}

To close this section we prove the following result which states that the $\infty$-transverse implications of an element in the cohomology of the total space of a fibration can be read in the cohomology of the fibre. 

\begin{lem}
Let $j:F\to X$ be a continuous map. If $x\in H^*(X;\F_2)$ is such that $j^*(x)\not=0\in H^*(F;\F_2)$ is $\infty$-transverse, then $x$ is itself $\infty$-transverse.
\end{lem}

\begin{proof} %[Proof of Lemma \ref{l:transversity from the fibre}]
Suppose that $x$ is not $\infty$-transverse. Then there exists $r\geq0$ such that $d_{r+1}x^{2^r}=0$. Therefore we have
\begin{align*}
d_{r+1}j^*(x)^{2^r} &=d_{r+1}j^*(x^{2^r}) &&\text{since $j^*$ is an algebra map,}\\
&=j^*d_{r+1}x^{2^r} &&\text{by naturality,}\\
&=0 &&\text{since $d_{r+1}x^{2^r}=0$},
\end{align*} which contradicts $\infty$-transversity of $j^*(x)$.
\end{proof}

\section{Examples}\label{s:examples}

\subsection{A space ``with retract''}\label{e:retract}

Let $X$ be the space given by the fibration
$$\xymatrix{
X\ar[r]^-i &K(\Z/2,2)\times K(\Z/2,2)\ar[r]^-k &K(\Z/2,4),
}$$
where its single non-trivial k-invariant is
\begin{align*}
k \in& [K(\Z/2,2)\times K(\Z/2,2),K(\Z/2,4)]\\
\cong &H^4(K(\Z/2,2)\times K(\Z/2,2);\F_2)\\
\cong &H^4(K(\Z/2,2);\F_2)\otimes\F_2\\
&\oplus H^2(K(\Z/2,2);\F_2)\otimes H^2(K(\Z/2,2);\F_2)\\
&\oplus\F_2\otimes H^4(K(\Z/2,2);\F_2)\\
\cong &\F_2\{u_2^2\otimes1,u_2\otimes v_2,1\otimes v_2^2\}
\end{align*}
given by $k=u_2\otimes v_2$ where $u_2$ and $v_2$ are the fundamental classes of both copies of $K(\Z/2,2)$. The space $X$ has only two non-trivial homotopy groups $\pi_2(X)\cong\Z/2\oplus\Z/2$ and $\pi_3(X)\cong\Z/2$.

\newpage
\begin{thm*}
The space $X$ has the following properties:
\begin{itemize}
\item[1.]{$X$ is not a GEM (i.e. a weak product of Eilenberg-Mac\,Lane spaces),}
\item[2.]{$X$ is not a H-space,}
\item[3.]{$X$ retracts (weakly) onto the Eilenberg-Mac\,Lane space $K(\Z/2,2)$, i.e. there exist maps $f:X\to K(\Z/2,2)$ and $g:K(\Z/2,2)\to X$ such that $fg\simeq\id_{K(\Z/2,2)}$,}
\item[4.]{$f^*:H^*(K(\Z/2,2);\F_2)\to H^*(X;\F_2)$ is a monomorphism,}
\item[5.]{$X$ has no homological exponent.}
\end{itemize}
\end{thm*}

\begin{proof}
The space $X$ is not a GEM since its k-invariant $u_2\otimes v_2$ is not trivial. Moreover, $u_2\otimes v_2$ is decomposable in $H^*(K(\Z/2,2)\times K(\Z/2,2);\F_2)$ and therefore not primitive. Thus $X$ is not a H-space. 

Consider the following homotopy commutative diagram based on the fibration for which $X$ is the fibre:
$$\xymatrix{
&X\ar[d]^i\ar@/^/[rd]^f\\
K(\Z/2,2)\ar[r]^-{i_1}\ar@/^/@{.>}[ru]^g\ar@/_/[rd]_{*} &K(\Z/2,2)\times K(\Z/2,2)\ar[r]^-{p_1}\ar[d]^k &K(\Z/2,2)\\
&K(\Z/2,4),
}$$ where $i_1$ denotes the inclusion into the first factor, $p_1$ denotes the projection onto the first factor and $f=p_1i$. The existence of a (generally not unique) map $g$ is a consequence of the fact that $k i_1\simeq*$. To see that $k i_1\simeq*$, recall first that the isomorphism $[K(\Z/2,2),K(\Z/2,4)]\cong H^4(K(\Z/2,2);\F_2)$ maps $ki_1$ to $(ki_1)^*(u_4)$, where $(ki_1)^*=(i_1)^*k^*:H^4(K(\Z/2,4);\F_2)\to H^4(K(\Z/2,2);\F_2)$ and $u_4\in H^4(K(\Z/2,4);\F_2)$ is the fundamental class. Now we have 
\begin{align*}
(i_1)^*k^*(u_4) &=(i_1)^*(u_2\otimes v_2) &&\text{by definition of $k$,}\\
&=(i_1)^*(u_2\otimes1\cdot1\otimes v_2)\\
&=(i_1)^*(u_2\otimes1)\cdot(i_1)^*(1\otimes v_2) &&\text{since $(i_1)^*$ is a ring map,}\\
&=0 &&\text{since $(i_1)^*(1\otimes v_2)=0$.}
\end{align*}
Therefore $fg\simeq p_1ig\simeq p_1i_1=\id$ i.e. $X$ retracts (weakly) onto $K(\Z/2,2)$. Consider now the following induced commutative diagram:
$$\xymatrix{
&H^*(X;\F_2)\ar@/^/[rd]^{g^*}\\
H^*(K(\Z/2,2);\F_2)\ar@/^/[ru]^{f^*}\ar@{=}[rr] &&H^*(K(\Z/2,2);\F_2).
}$$ The induced map $f^*$ is clearly a monomorphism. Let us remark that this remains true if one look at the induced diagram via $H^*(-;R)$ for any ring $R$. The fact that $K(\Z/2,2)$ has no homology exponent then obviously implies that the same is true for $X$.
\end{proof}

\newpage
One can relate this space with other examples pointed out by F. R. Cohen and F. P. Peterson in \cite{CP00}. They constructed loop maps $\Omega g:\Omega Y\to K(\Z/2,n)$, $n\geq2$, with the property that $(\Omega g)^*:H^*(K(\Z/2,n);\F_2)\to H^*(\Omega Y;\F_2)$ is a monomorphism and such that $\Omega g$ does not admit a section. 
%In particular, using the classical result on the structure of Hopf algebras that a connected Hopf algebra is free over a sub-Hopf algebra, they proved that $H^*(\Omega X;\F_2)$ is a free module over $H^*(K(\Z/2,n);\F_2)$. 

The examples are mainly provided by $\Omega\Sigma(\R P^\infty)^n\to K(\Z/2,n)$, the canonical multiplicative extension of Serre's map $e:(\R P^\infty)^n\to K(\Z/2,n)$, and by $\Omega\Sigma BSO(3)\to K(\Z/2,2)$, the canonical multiplicative extension of the second Stiefel-Whitney class in the mod-$2$ cohomology of $BSO(3)$ in the case $n=2$.

The spaces $\Sigma(\R P^\infty)^n$ and $\Sigma BSO(3)$ have infinitely many non-trivial homotopy groups. Our space $X$ has only two. However, the loop maps $\Omega g:\Omega Y\to K(\Z/2,n)$ of F. R. Cohen and F. P. Peterson and our map $f:X\to K(\Z/2,2)$ all induce monomorphisms.

\subsection{A space ``without retract''}\label{e:no_retract}

Let us now consider another interesting example of space with two non-trivial homotopy groups. As the title of this paragraph suggests, the space we are about to construct will not admit a retract. Therefore we will not be able to use this topological feature in order to prove the non-existence of a homology exponent. The main idea here is to detect $\infty$-transverse implications in our example, and full usage of the material exposed in the previous section will be made.

Let $X$ be the space given by the fibration
$$\xymatrix{
X\ar[r]^-i &K(\Z/2,2)\ar[r]^-k &K(\Z/2,4),
}$$
where its single non-trivial k-invariant is
\begin{align*}
k \in& [K(\Z/2,2),K(\Z/2,4)]\\
\cong &H^4(K(\Z/2,2);\F_2)\\
\cong &\F_2\{u_2^2\}
\end{align*}
given by $k=u_2^2$ where $u_2$ is the fundamental class of $K(\Z/2,2)$. The space $X$ has only two non-trivial homotopy groups $\pi_2(X)\cong\Z/2\cong\pi_3(X)$. Moreover, this is a H-space.

\begin{thm*}
The space $X$ has the following properties:
\begin{itemize}
\item[1.]{$X$ is not a GEM,}
\item[2.]{$X$ is an infinite loop space,}
\item[3.]{$X$ retracts neither onto the Eilenberg-Mac\,Lane space $K(\Z/2,2)$, nor onto $K(\Z/2,3)$,}
\item[4.]{However, $X$ has no homology exponent.}
\end{itemize}
\end{thm*}

\begin{proof}
The space $X$ is not a GEM since its k-invariant $u_2^2$ is not trivial. It is an infinite loop space since $u_2^2=Sq^2u_2=\sigma^*Sq^2u_3=\sigma^{(2)}Sq^2u_4=\dots$, where $\sigma^{(n)}$ denotes the $n$-fold cohomology suspension. 

\newpage
In order to show that $X$ does retract neither onto $K(\Z/2,2)$, nor onto $K(\Z/2,3)$, let us consider the mod-$2$ cohomology Serre spectral sequence of the fibration $\xymatrix{K(\Z/2,3)\ar[r]^-j &X\ar[r]^-i &K(\Z/2,2)}$. 
The $E_2$-term looks like the following:
$$\xymatrix@R=0.1cm@C=0.1cm{
&&\\
{\bf Sq^1u_3}  &&0 &{*} &{*} &{*} &{*} &{*}\\
{\bf u_3}  &&0 &u_2u_3&{*} &{*} &{*} &{*}\\
{\bf 0} &&0 &0 &0 &0 &0 &0\\
{\bf 0} &&0 &0 &0 &0 &0 &0\\ \ar@{-}[rrrrrrrr] &&&&&&&&\\
{\bf 1} &\ar@{-}[uuuuuu] &{\bf 0} &{\bf u_2} &{\bf Sq^1u_2} &{\bf u_2^2} &{\bf Sq^{2,1}u_2} &{\bf u_2^3}\\
&&&&&&{\bf u_2Sq^1u_2} &{\bf (Sq^1 u_2)^2}
}$$
We have $\bigoplus_{s}E_\infty^{s,2-s}\cong E_\infty^{2,0}$ and $H^2(X;\F_2)\cong\F_2\{v\}$ with $u_2\mapsto v$ via the composition
$$
i^*:\xymatrix{H^2(K(\Z/2,2);\F_2)\cong E^{2,0}_2\ar@{->>}[r] &E^{2,0}_3\cong E^{2,0}_\infty\cong H^2(X;\F_2)}.
$$ 
The transgression on $u_3$ is given by the k-invariant. To see this, consider the following homotopy pullback along the path-loop fibration:
$$\xymatrix{
K(\Z/2,3)\ar[d]_j\ar@{=}[r] &K(\Z/2,3)\ar[d]\\
X\ar[d]_i\ar[r] &{*}\ar[d]\\
K(\Z/2,2)\ar[r]_-k &K(\Z/2,4)
}$$ and the Serre spectral sequence of both columns. By naturality of the spectral sequence, we have the following commutative diagram:
$$\xymatrix@C=2truecm{
H^3(K(\Z/2,3);\F_2)\ar@{=}[d]\ar[r]^-\cong &H^4(K(\Z/2,4);\F_2)\ar[d]^{k^*}\\
H^3(K(\Z/2,3);\F_2)\ar[r]_-{\text{(transgression)}} &H^4(K(\Z/2,2);\F_2).
}$$ The Serre's transgression theorem (see for instance \cite[Theorem 6.8, p. 189]{Mc00}) implies that $d_4$ coincides with the transgression. Thus we have $d_4u_3=k^*(u_4)=u_2^2$. 

Therefore $\bigoplus_{s}E_\infty^{s,3-s}\cong E_\infty^{3,0}$ and $H^3(X;\F_2)\cong\F_2\{w\}$ with $Sq^1u_2\mapsto w$ via the composition
$$
i^*:\xymatrix{H^*(K(\Z/2,2);\F_2)\cong E_2^{3,0}\ar@{->>}[r] &E_3^{3,0}\ar@{->>}[r] &E_4^{3,0}\cong E_\infty^{3,0}\cong H^3(X;\F_2)}.
$$ 
We clearly have $Sq^1v=w$.

Suppose that there are maps $f:X\to K(\Z/2,2)$ and $g:K(\Z/2,2)\to X$ with $fg\simeq \id_{K(\Z/2,2)}$. The only non-trivial map $f:X\to K(\Z/2,2)$ is given by the single non-trivial element $v\in H^2(X;\F_2)$. This forces $f\simeq i$. Therefore we have $kig\simeq k\id_{K(\Z/2,2)}\simeq k$ which contradicts the fact that $ki\simeq *$.

Suppose now that there are maps $f:X\to K(\Z/2,3)$ and $g:K(\Z/2,3)\to X$ with $fg\simeq \id_{K(\Z/2,3)}$. The only non-trivial map $f:X\to K(\Z/2,3)$ is given by the single non-trivial element $w\in H^3(X;\F_2)$. Therefore we have \begin{align*}
g^*f^*(u_3)&=g^*(w)\\
&=g^*(Sq^1v)\\
&=Sq^1g^*(v) &&\text{by naturality,}\\
&=0 &&\text{since $g^*(v)\in H^2(K(\Z/2,3);\F_2)=0$.}
\end{align*} In other words, we always have $fg\simeq*$ and $X$ cannot retract onto $K(\Z/2,3)$.

%Let us show that $Sq^1u_3\not=0\in E_5^{0,4}$. For connexity reasons, it suffices to show that $d_2Sq^1u_3=0\in E_2^{2,3}=\F_2\{u_2u_3\}$. Suppose that $d_2Sq^1u_3=u_2u_3$. Then we would have $u_2u_3=0\in E_3^{2,3}$. This is absurd since $u_2\not=0\in E_3^{2,0}$ and $u_3\not=0\in E_3^{0,3}$ as well as their product $u_2u_3\not=0\in E_3^{2,3}$. 
%
%Let us show now that $d_5Sq^1u_3=0$. Since cohomology operations ``commute'' with transgressions (see \cite[Corollary 6.9, p. 189]{Mc00}), we have $d_5Sq^1u_3=Sq^1 u_2^2=0$. 
%
%Finally, we conclude that $Sq^1u_3\not=0\in E_6^{0,4}\cong E_\infty^{0,4}$. 

Since cohomology operations ``commute'' with transgressions (see \cite{Si73-I} for the very precise meaning of this assertion), and since $u_3$ transgresses to $u_2^2$, $Sq^1u_3$ transgresses to $Sq^1u_2^2=0$ by Cartan's formula. Therefore $Sq^1u_3\not=0\in E_\infty^{0,4}$ and there exists $x'\in H^4(X;\F_2)$ such that $x'\mapsto Sq^1u_3$ via the composition
$$
j^*:\xymatrix{H^4(X;\F_2)\ar@{->>}[r] &E^{0,4}_\infty\cong E^{0,4}_6\subset\dots\subset E^{0,4}_2\cong H^4(K(\Z/2,3);\F_2)}.
$$ Set $x=Sq^2x'$. We have $j^*(x)=j^*(Sq^2x')=Sq^2j^*(x')=Sq^{2,1}u_3$ which is $\infty$-transverse. Thus $x$ is also $\infty$-transverse and $X$ cannot admit a homology exponent. 
\end{proof}

\subsection{On the diversity}

The first example, which possesses a retract, is not a H-space since its single k-invariant is not a H-map. The second example, which admits no retract, is a H-space. Therefore one can ask whether it is a general property for non-H-spaces to admit a retract and for H-spaces not to admit any. We show here that there is no correlation between these two features and that actually all cases occur.

Let us consider any product of Eilenberg-Mac\,Lane spaces, so called generalized Eilenberg-Mac\,Lane spaces or GEM. They obviously admit a H-space structure, although it is not unique in general, and they also admit retracts. Another less trivial example is given by the homotopy fibre of the application 
$$\xymatrix{K(\Z/2,2)\times K(\Z/2,2)\ar[r]^-{u_2^2\otimes1}&K(\Z/2,4)},$$
where $u_2$ is the characteristic class of $K(\Z/2,2)$. It is not difficult to verify that this space admits a retract onto the second copy $K(\Z/2,2)$ and that it is a H-space.

Finally, consider the space $X$ given by the homotopy fibre of the following application:
$$\xymatrix@C=3cm{
K(\Z/2,2)\times K(\Z/2,2)\ar[r]^-{u_2^2\otimes1+u_2\otimes v_2+1\otimes v_2^2} &K(\Z/2,4),
}$$ where $u_2$ and $v_2$ are the two characteristic classes of each copies of $K(\Z/2,2)$. It is easy to show that this space admits no retract and that it is not a H-space.

The following table summarize all the cases which we encountered:
\begin{center}
\begin{tabular}{|r|c|c|}
\hline
&with retract &without retract\\
\hline
\hfill H-space &any GEM &example \ref{e:no_retract}\\
\hline
non-H-space &example \ref{e:retract} &$X$\\
\hline
\end{tabular}
\end{center}

%\begin{center}
%\begin{tabular}{p{3cm}p{6cm}}
%&\begin{tabular}{|p{3cm}|p{3cm}|}
%\hline
%with retract &without retract\\
%\hline
%\end{tabular}\\
%\begin{tabular}{|p{3cm}|}
%\hline
%H-space\\
%\hline
%non-H-space\\
%\hline
%\end{tabular}
%&\begin{tabular}{|p{3cm}|p{3cm}|}
%\hline
%any GEM &example \ref{e:no_retract}\\
%\hline
%example \ref{e:retract} &$X$\\
%\hline
%\end{tabular}
%\end{tabular}
%\end{center}

%\subsection{A space as a ``prototype''}

%Let us now consider another example of a space with two non-trivial homotopy groups. Let $s,t\geq1$ and $X$ be the space given by the fibration
%$$\xymatrix{
%X\ar[r]^-i &K(\Z/2^s,2)\ar[r]^-k &K(\Z/2^t,4),
%}$$
%where its single non-trivial k-invariant is required to be a H-map and lies in
%\begin{align*}
%k \in& [K(\Z/2^s,2),K(\Z/2^t,4)]\\
%\cong &H^4(K(\Z/2^s,2);\Z/2^t)\\
%\cong &\Hom(H_4(K(\Z/2^s,2);\Z),\Z/2^t)\\
%\cong &\Hom(\Z/2^{s+1},\Z/2^t)\\
%\cong &\Z/2^{\min(s+1,t)}.
%\end{align*}
%The space $X$ has only two non-trivial homotopy groups $\pi_2(X)\cong\Z/2^s$ and $\pi_3(X)\cong\Z/2^t$ and it is a H-space.
%
%\begin{thm*}
%The space $X$ has no homology exponent.
%\end{thm*}
%
%\begin{proof}
%Let us consider the mod-$2$ cohomology Serre spectral sequence of the fibration $\xymatrix{K(\Z/2^t,3)\ar[r]^-j &X\ar[r]^-i &K(\Z/2^s,2)}$. 
%
%The $E_2$-term looks like the following:
%$$\xymatrix@R=0.1cm@C=0.1cm{
%&&\\
%{\bf \delta_t\iota_3}  &&0 &{*} &{*} &{*} &{*} &{*}\\
%{\bf u_3}  &&0 &u_2u_3&{*} &{*} &{*} &{*}\\
%{\bf 0} &&0 &0 &0 &0 &0 &0\\
%{\bf 0} &&0 &0 &0 &0 &0 &0\\ \ar@{-}[rrrrrrrr] &&&&&&&&\\
%{\bf 1} &\ar@{-}[uuuuuu] &{\bf 0} &{\bf u_2} &{\bf \delta_s\iota_2} &{\bf u_2^2} &{\bf Sq^2\delta_s\iota_2} &{\bf u_2^3}\\
%&&&&&&{\bf u_2\delta_s\iota_2} &{\bf (\delta_s\iota_2)^2}
%}$$

%Let us show that $\delta_t\iota_3\in E_5^{0,4}$ is transgressive. For connexity reasons, it suffices to show that $d_2\delta_t\iota_3=0\in E_2^{2,3}=\F_2\{u_2u_3\}$. Suppose that $d_2\delta_t\iota_3=u_2u_3$. Then we would have $u_2u_3=0\in E_3^{2,3}$. This is absurd since $u_2\not=0\in E_3^{2,0}$ and $u_3\not=0\in E_3^{0,3}$ as well as their product $u_2u_3\not=0\in E_3^{2,3}$. 

%We then have
%\begin{align*}
%d_5\delta_t\iota_3 &= \delta_t d_4 \iota_3 &&\text{see \cite[p. 206]{Se53} and \cite[p. 457]{Se51},}\\
%&=\delta_t k^*\iota_3 &&\text{since $d_4$ is given by the k-invariant,}\\
%&=k^*\delta_t\iota_3 &&\text{by naturality,}
%\end{align*} which is a primitive element since $k$ is a H-map and thus $d_5\delta_t\iota_3\F_2\{Sq^2\delta_s\iota_2\}$. Suppose that $d_5\delta_t\iota_3=Sq^2\delta_s\iota_2$, we then have 
%\begin{align*}
%d_7Sq^2\delta_t\iota_3 &= Sq^2d_5\delta_t\iota_3 &&\text{since $Sq^2$ commutes with the transgressions,}\\
%&= Sq^2Sq^2\delta_s\iota_2\\
%&= Sq^3Sq^1\delta_s\iota_2 &&\text{by Adem relations,}\\
%&= 0 &&\text{since $Sq^1\delta_s=0$ for all $s\geq1$.}
%\end{align*} In all cases, $Sq^2\delta_t\iota_3\not=0\in E_\infty^{0,6}$. Let $x\in H^6(X;\F_2)$ such that $j^*(x)=Sq^2\delta_t\iota_3$. The element $Sq^2\delta_t\iota_3$ is $\infty$-transverse and so is $x$.
%\end{proof}

%\subsection{First two non-trivial homotopy groups}

%Let us now consider all H-spaces with the first two non-trivial homotopy groups. Let $G\cong\oplus_a\Z/2^{s_a}$ and $H\cong\oplus_b\Z/2^{t_b}$ be two $2$-torsion finite groups and $X$ be the space given by the fibration
%$$\xymatrix{
%X\ar[r]^-i &K(G,2)\ar[r]^-k &K(H,4),
%}$$
%where its single non-trivial k-invariant is a H-map. The space $X$ has only two non-trivial homotopy groups $\pi_2(X)\cong G$ and $\pi_3(X)\cong H$ and it is a H-space.
%
%\begin{thm*}
%The space $X$ has no homology exponent.
%\end{thm*}
%
%\begin{proof}
%Let us consider the mod-$2$ cohomology Serre spectral sequence of the fibration $\xymatrix{K(H,3)\ar[r]^-j &X\ar[r]^-i &K(G,2)}$. 

%Let $i_2\in H^2(K(G,2);G)$ be the characteristic class and consider the elements $\iota_{2,a}\in H^2(K(G,2);\Z/2^{s_a})$ induced by the projections of $G$ on all the factors $\Z/2^{s_a}$. Consider also all the $u_{2,a}\in H^2(K(G,2);\F_2)$ given by the reduction mod $2$ of the $\iota_{2,a}$'s.

%Analogously, let $j_3\in H^3(K(H,3);H)$ be the characteristic class and consider the elements $\jmath_{3,b}\in H^3(K(H,3);\Z/2^{t_b})$ induced by the projections of $H$ on all the factors $\Z/2^{t_b}$. Consider also all the $v_{3,b}\in H^3(K(H,3);\F_2)$ given by the reduction mod $2$ of the $\jmath_{3,b}$'s. Moreover, pick $(t,v_3,\jmath_3)$ among $\{(t_b,v_{3,b},\jmath_{3,b})\ |\ \text{for all $b$}\}$.
%
%The $E_2$-term looks like the following:
%$$\xymatrix@R=0.1cm@C=0.1cm{
%&&\\
%{\bf \delta_t\jmath_3,*}  &&0 &{*} &{*} &{*} &{*} &{*}\\
%{\bf v_3,*}  &&0 &{*} &{*} &{*} &{*} &{*}\\
%{\bf 0} &&0 &0 &0 &0 &0 &0\\
%{\bf 0} &&0 &0 &0 &0 &0 &0\\ \ar@{-}[rrrrrrrr] &&&&&&&&\\
%{\bf 1} &\ar@{-}[uuuuuu] &{\bf 0} &{*} &{*} &{*} &{*} &{*}}
%$$

%Let us show that $\delta_t\jmath_3\in E_5^{0,4}$ is transgressive. For connexity reasons, it suffices to show that $d_2\delta_t\jmath_3=0\in E_2^{2,3}\cong\F_2\{u_{2,a}v_{3,b}\ |\ \text{for all $a$ and $b$}\}$. 
%Suppose that $d_2\delta_t\jmath_3=\sum_{(\alpha,\beta)} u_{2,\alpha}v_{3,\beta}$. Then we would have $\sum_{(\alpha,\beta)} u_{2,\alpha}v_{3,\beta}=0\in E_3^{2,3}$. This is absurd since $u_{2,\alpha}\not=0\in E_3^{2,0}$ and $v_{3,\beta}\not=0\in E_3^{0,3}$, as well as $\sum_{(\alpha,\beta)} u_{2,\alpha}v_{3,\beta}$. 

%Since $k$ is a H-map, $\delta_t\jmath_3$ transgresses to a primitive element which lies in $$\F_2\{Sq^2\delta_{s_a}\iota_{2,a}\ |\ \text{for all $a$}\}.$$ Suppose that $d_5\delta_t\jmath_3=\sum_{\alpha}Sq^2\delta_{s_\alpha}\iota_{2,\alpha}$, we then have 
%\begin{align*}
%d_7Sq^2\delta_t\jmath_3 &= Sq^2d_5\delta_t\jmath_3 &&\text{since $Sq^2$ commutes with the transgressions,}\\
%&= \sum_{\alpha}Sq^2Sq^2\delta_{s_\alpha}\iota_{2,\alpha}\\
%&= \sum_{\alpha}Sq^3Sq^1\delta_{s_\alpha}\iota_{2,\alpha} &&\text{by Adem relations,}\\
%&= 0 &&\text{since $Sq^1\delta_s=0$ for all $s\geq1$.}
%\end{align*} In all cases, $Sq^2\delta_t\jmath_3\not=0\in E_\infty^{0,6}$. Let $x\in H^6(X;\F_2)$ such that $j^*(x)=Sq^2\delta_t\jmath_3$. The element $Sq^2\delta_t\jmath_3$ is $\infty$-transverse and so is $x$.
%\end{proof}

\section{Proof of the main result}\label{s:proofs}

This section is devoted to the proof of our main result. We begin to sketch a strategy allowing us to reach our aim.

\subsection*{Strategy}
Our example ``without retract'' permits us to release a general strategy aiming to establish the non-existence of a homology exponent. Let us outline what happened in this particular example. We found an element $x$ in the mod-$2$ cohomology of $X$ such that its image in the fibre $j^*(x)=Sq^{2,1}u_3$ is not only (A) non-trivial but also (B) has $\infty$-transverse implications. Moreover, we showed that this imply for $x$ itself to have $\infty$-transverse implications, too. This obviously prevents the existence of a homology exponent. Let us look at each of these stages more closely and technically:
\begin{itemize}
\item[(A)]{Every non-trivial element in $E_2^{0,*}$ which survives to $E_\infty^{0,*}$ gives rise to an element $x\in H^*(X;\F_2)$ with a non-trivial image $j^*(x)$ in the cohomology of the fibre. This leads us to find an element in $E_2^{0,*}$ which transgresses to zero. The fact that transgressions in the spectral sequence commute with the Steenrod squares is very valuable for our purpose (J.-P. Serre remarked that the transgressions also commute with the operations $\delta_s$, see \cite[p. 206]{Se53} and \cite[p. 457]{Se51}). For instance, the reduction of the characteristic class $u_n\in E_2^{0,n}$ transgresses to a primitive element $w$ determined by the k-invariant. Therefore, we are done if we can find an admissible sequence $I$ such that $Sq^I_s u_n\not=0$ transgresses to $Sq^I_s w=0$. This relies on the $\A_2$-action on the primitive elements of $E_2^{*,0}$.}
\item[(B)]{Among all the admissible sequences $I=(a_0,\dots)$ of excess $e(I)<n$ such that $Sq^I_s u_n\not=0$ transgresses to zero, some of them are interesting because they insure on one hand that the element $Sq^I_s u_n$ has a $0$-transverse implication in the cohomology of the fibre and on the other hand that $Sq^I_s u_n$ lies in even degree. We have seen that these two conditions force $Sq^I_s u_n$ to have $\infty$-implications. It is immediate to see that such an admissible sequence ``begins'' with $a_0$ even and has a stable degree $\degst(I)$ of the same parity than $n$.}
\end{itemize}

Let us look at the $\A_2$-action on the primitive elements of the cohomology of the base space. Following (A), our aim here is to find a suitable admissible sequence. Consider the following definition:

\begin{defn}
For all $l\geq1$ define the admissible sequence $\xi(l)=(2^l+2^{l-1}-1,\dots,5,2)$ of stable degree $\degst(\xi(l))=2^{l+1}+2^l-l-3$ and excess $e(\xi(l))=l+1$.
\end{defn}

\begin{prop}
Let $m\geq2$ and let $G$ be a $2$-torsion finite abelian group and consider the $\A_2$-module of primitives $P^*H^*(K(G,m);\F_2)$. Let $n\geq m+1$. Then we have
$$
\begin{cases}
Sq^2P^5H^*(K(G,2);\F_2)=0.\\
Sq^3P^6H^*(K(G,3);\F_2)=0.\\
Sq^{3}P^{n+1}H^*(K(G,m);\F_2)=0 &\text{if $n=m+1\geq4$.}\\
Sq^{\xi(n-3)}P^{n+1}(K(G,m);\F_2)=0 &\text{if $n\geq m+2$.}
\end{cases}
$$
\end{prop}

\newpage
\begin{proof}
Since $G$ is a $2$-torsion finite abelian group, we can write $G\cong\oplus_a\Z/2^{s_a}$. 

Suppose that $n=m+1=3$. Let $i_2\in H^2(K(G,2);G)$ be the characteristic class and consider the elements $\iota_{2,a}\in H^2(K(G,2);\Z/2^{s_a})$ induced by the projections of $G$ on each of the factors $\Z/2^{s_a}$. Let $x\in P^5H^*(K(G,2);\F_2)$. There obviously exists $y\in H^3(K(G,2);\F_2)$ of the form $y=\sum\delta_s\iota_{2,s}$ and such that $x=Sq^2y$. We have 
\begin{align*}
Sq^2 x &=Sq^2 Sq^2y\\
&=Sq^{3,1}y &\text{by Adem relations,}\\
&=\sum Sq^{3,1}\delta_s\iota_{2,s}=0 &\text{since $Sq^1\delta_s=0$ for all $s$.}
\end{align*}

Suppose that $n=m+1=4$ and let $x\in P^6H^*(K(G,3);\F_2)$. There exists $y\in H^3(K(G,3);\F_2)$ such that $x=y^2$. We then have $Sq^3x=Sq^3y^2=0$ by Cartan's formula.

Suppose that $n=m+1\geq4$ and let $x\in P^{m+2}H^*(K(G,m);\F_2)$. There exists $y\in H^m(K(G,m);\F_2)$ such that $x=Sq^2y$. We then have $Sq^3 x=Sq^3Sq^2y=0$ by Adem relations.

Finally suppose that $n\geq m+2$. For all $m\geq2$ define the following subsets of the integers:
\begin{align*}
M^m &=\{2^i+2^{i-1}\ |\ \text{for all $i\geq m$}\} \text{ and}\\
N^m &=\{1+2^{h_1}+\dots+2^{h_{m-1}}\ |\ h_1\geq\dots\geq h_{m-1}\geq0\}.
\end{align*}
It is a very simple arithmetic game to see that $M^m\cap N^m=\emptyset$. Careful calculations show that for all admissible sequence $I$, we have $e(I)<m$ if and only if $\degst(I)+m\in N^m$. Thus there is no admissible sequence $I$ of excess $e(I)<m$ such that $\degst(I)+m=2^{\geq m}+2^{\geq m-1}$. Therefore
$$
Q^{2^{\geq m}+2^{\geq m-1}}H^*(K(G,m);\F_2)=0.
$$
Let $x\in P^{n+1}H^*(K(G,m);\F_2)$. We have
\begin{align*}
\deg(Sq^{2^{n-3}+2^{n-4}-2}Sq^{\xi(n-4)}x) = &\ \deg(x) + \degst(\xi(n-4)) + (2^{n-3}+2^{n-4}-2)\\
= &\ (n+1) + (2^{n-3}+2^{n-4}-(n-4)-3)\\ 
&\ +(2^{n-3}+2^{n-4}-2)\\
= &\ 2^{n-2}+2^{n-3}.
\end{align*}
Thus $Sq^{2^{n-3}+2^{n-4}-2}Sq^{\xi(n-4)}x$ is decomposable since $Q^{2^{n-2}+2^{n-3}}H^*(K(G,m);\F_2)=0$ when $n\geq m+2$, and it is a square (maybe zero) since it is primitive. Therefore 
\begin{align*}
Sq^{\xi(n-3)}x &= Sq^{2^{n-3}+2^{n-4}-1}Sq^{\xi(n-4)}x\\
&=Sq^1Sq^{2^{n-3}+2^{n-4}-2}Sq^{\xi(n-4)}x\\
&=Sq^1(\text{square}) =0.
\end{align*}

\end{proof}

Let $X$ be a non-contractible simply-connected $2$-local H-space with exactly two non-trivial homotopy groups $\pi_m(X)\cong G$ and $\pi_n(X)\cong H$, where $n>m\geq2$ and $G\cong\oplus_a\Z/2^{s_a}$, $H\cong\oplus_b\Z/2^{t_b}$ are finite groups. 

\newpage
The space $X$ fits into the fibrations
$$\xymatrix{
K(H,n)\ar[r]^-j &X\ar[r]^-i &K(G,m)\ar[r]^-k &K(H,n+1),
}$$
where $k$ is the single k-invariant of the space $X$. Since $X$ is a H-space, $k$ is a H-map.

%If $m=2$ and $n=3$, this is essentially the example without retract of the preceding section. It is not difficult to adapt the proof in the more general context we are dealing with. Actually, it suffices to show that the operations $\delta_s$ also commute with the transgressions in the Serre spectral sequence (see \cite[p.206]{Se53}).

Let $i_m\in H^2(K(G,m);G)$ be the characteristic class and consider the elements $\iota_{m,a}\in H^m(K(G,m);\Z/2^{s_a})$ induced by the projections of $G$ on all the factors $\Z/2^{s_a}$. Consider also all the $u_{m,a}\in H^m(K(G,m);\F_2)$ given by the reduction mod $2$ of the $\iota_{m,a}$'s.

Analogously, let $j_n\in H^n(K(H,n);H)$ be the characteristic class and consider the elements $\jmath_{n,b}\in H^n(K(H,n);\Z/2^{t_b})$ induced by the projections of $H$ on all the factors $\Z/2^{t_b}$. Consider also all the $v_{n,b}\in H^n(K(H,n);\F_2)$ given by the reduction mod $2$ of the $\jmath_{n,b}$'s. Moreover, pick $(t,v_n,\jmath_n)$ among $\{(t_b,v_{n,b},\jmath_{n,b})\ |\ \text{for all $b$}\}$.

The $E_2$-term looks like the following:
$$\xymatrix@R=0.1cm@C=0.1cm{
&&\\
{\bf \delta_t\jmath_n,*}  &&0 &\dots &0 &{*} &{*} &{*}\\
{\bf v_n,*}  &&0 &\dots &0 &{*} &{*} &{*}\\
{\bf 0} &&0 &\dots &0 &0 &0 &0\\
{\bf \vdots} &&\vdots & &\vdots &\vdots &\vdots &\vdots\\
{\bf 0} &&0 &\dots &0 &0 &0 &0\\ \ar@{-}[rrrrrrrr] &&&&&&&&\\
{\bf 1} &\ar@{-}[uuuuuu] &{\bf 0} &{\dots} &{0} &{*} &{*} &{*}}
$$

The element $v_n$ transgresses to $d_{n+1}v_n$ in $P^{n+1}H^*(K(G,m);\F_2)$ which is determined by the k-invariant. Set 
$$
\xi=\begin{cases}
(2,1) &\text{if $m=2$ and $n=3$,}\\
(6,3,1) &\text{if $m=3$ and $n=4$,}\\
(6,3) &\text{if $n=m+1\geq5$ and $n$ is odd,}\\
(14,7,3) &\text{if $n=m+1\geq6$ and $n$ is even,}\\
(2^{n-2}+2^{n-3}-2,\xi(n-3)) &\text{if $n\geq m+2\geq4$.}\\
\end{cases}
$$

In all cases $e(\xi)<n$, $\deg(Sq^\xi_t v_n)$ is even and $Sq_{t}^\xi v_n$ transgresses to $Sq_t^\xi d_{n+1}v_n$ which is trivial by the preceding proposition. Let $x\in H^*(X;\F_2)$ such that $j^*(x)=Sq_t^\xi v_n$. The element $Sq_t^\xi v_n$ is $\infty$-transverse and so is $x$.

\section{Generalizations}\label{s:generalizations}

We conclude this paper with some possible generalizations of our main result.

The first generalization that we are interested in concerns the nature of the two non-trivial homotopy groups. We supposed them to be finite. Let us suppose now that the homotopy groups are of finite type. Copies of $\Z_{(2)}$, which have no torsion, may appear in this extended context. The cohomology of Eilenberg-Mac\,Lane spaces associated to such groups was also computed by H. Cartan and J.-P. Serre:
$$
H^*(K(\Z_{(2)},n);\F_2)\cong\F_2[Sq^Iu_n\ |\ \text{$I=(a_0,...,a_k)$ with $a_k\not=1$ and $e(I)<n$}].
$$
We say that a space $X$ admits a {\bf torsion homology exponent} if there exists an exponent for the torsion subgroup of $H^*(X;\Z)$. With this definition, we have the following result on Eilenberg-Mac\,Lane spaces:

\begin{prop*}
Let $G$ be a non-trivial $2$-torsion abelian group of finite type and $n\geq4$. The Eilenberg-Mac\,Lane space $K(G,n)$ has no torsion homology exponent.
\end{prop*}

\begin{proof}
By K\"unneth formula and our preceding results when $G$ is finite, it is sufficient to suppose $G=\Z_{(2)}$. Consider the reduction of the fundamental class $u_n\in H^n(K(\Z_{(2)},n);\F_2)$. If $n$ is even, then $Sq^{2}u_n$ is $\infty$-transverse. If $n$ is odd, then $Sq^{6,3}u_n$ is $\infty$-transverse. 
\end{proof}

It is not very difficult using results of H. Cartan in \cite{Ca55} to verify that $K(\Z_{(2)},2)$ and $K(\Z_{(2)},3)$ admit torsion homology exponents. So the result is the best possible in terms of connexity. The following result is an extension to spaces with at most two non-trivial homotopy groups of finite type:

\begin{thm*}
Let $X$ be a non-contractible 3-connected $2$-local H-space of finite type with at most two non-trivial homotopy groups. Then $X$ has no torsion homology exponent.
\end{thm*}

The strategy and the admissible sequences $\xi$ listed in the proof of our main result are also convenient here to prove this theorem. 

\bigskip
W. Browder proved in \cite[Theorem 6.11, p. 46]{Br61} that every H-space of finite type which has the homotopy type of a finite CW-complex and which is simply-connected is actually $2$-connected. His result relies on the notion of $\infty$-implications in the Bockstein spectral sequence of the H-space. One can readilly check that an element which is $\infty$-transverse has $\infty$-implications. Therefore, it is interesting to set the following question. 

\begin{quest*}
Let $X$ be a simply-connected $2$-local H-space of finite type with a homology exponent. Is $X$ always $2$-connected? If it is not the case for all such H-spaces, is it true for infinite loop spaces?
\end{quest*}

\bigskip
In his PhD thesis which was published in \cite{Le95}, R. Levi studied the homotopy type of $p$-completed classifying spaces of the form $BG^\wedge_p$ for $G$ a finite $p$-perfect group, $p$ a prime. He constructed an algebraic analogue of Quillen's ``plus'' construction for differential graded coalgebras. He then proved that the loop spaces $\Omega BG^\wedge_p$ admit integral homology exponents. More precisely, he proved the following result:
\begin{thm*}
Let $G$ be a finite $p$-perfect group of order $p^r\cdot m$, $m$ prime to $p$. Then
$$
p^r\cdot\widetilde{H}_*(\Omega BG^\wedge_p;\Z_{(p)})=0.
$$
\end{thm*}
Moreover he proved that in general $BG^\wedge_p$ admits infinitely many non-trivial k-invariants, and thus in particular $\pi_*BG^\wedge_p$ is non-trivial in arbitrarily high dimensions. His method for proving this last result is based on a version of H. Miller's theorem improved by J. Lannes and L. Schwartz \cite{LS86}. This result and the results of the paper lead to the following conjecture:

\begin{conj*}
Let $X$ be a connected space. If $X$ has a homology exponent, then either $X\simeq K(\pi_1(X),1)$, or $X$ has infinitely many non-trivial k-invariants and, in particular, $X$ has infinitely many non-trivial homotopy groups. 
\end{conj*}

One way to attack this conjecture is to look first at the following problem at the prime $2$:

\begin{quest*}
Let $X$ be a $2$-local space (of finite type) and $G$ a finite $2$-torsion abelian group. If $X$ has a homology exponent, is the space $\map_*(K(G,2),X)$ weakly contractible?
\end{quest*}

To see that an affirmative answer to this question implies our conjecture at the prime $2$, suppose that $X$ is a $2$-local space with finitely many non-trivial k-invariants. Then $X\simeq X[m]\times\text{GEM}$ for some integer $m$. However, the fact that $X$ has no homology exponent implies that $X$ should be a Postnikov piece $X\simeq X[m]$. Consider then the Postnikov tower of the space $X[m]$:
$$\xymatrix{
K(\pi_mX,m)\ar[r]^-j &X[m]\ar[d]\\
&X[m-1]\ar[d]\ar[r]^-{k^{m+1}} &K(\pi_mX,m+1)\\
&\vdots\ar[d]\\
%&X[2]\ar[d]\\
&K(\pi_1X,1).
}$$ The map $j:K(\pi_m X,m)\to X[m]$ induces an isomorphism on the $m$-th homotopy groups. Therefore $\Omega^{m-2}j:K(\pi_m X,2)\to\Omega^{m-2}X[m]$ also induces an isomorphism on the second homotopy groups and thus the adjoint map $\Sigma^{m-2}K(\pi_m X,2)\to X[m]$ is not nullhomotopic. This contradicts the fact that $\map_*(K(\pi_mX,2),X[m])$ is weakly contractible.

\bigskip
\input biblio.tex

\end{document}